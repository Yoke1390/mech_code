\documentclass{jsarticle}
\usepackage{amsmath}
\usepackage[dvipdfmx]{graphicx}
\usepackage[dvipdfmx]{color}
\usepackage{mathtools}
\usepackage{physics}
\usepackage{siunitx}
\usepackage{wrapfig}
\usepackage{bm}
\mathtoolsset{showonlyrefs=true}

\begin{document}

\title{放射伝熱レポート}
\author{前田陽祐(03-240236)}
\maketitle
\section{炉内の金属線}
\section{2枚の遮蔽板}
無限に広い平行な灰色平面の熱流束の式は以下のようになる。
遮蔽板は十分に薄く熱容量を無視できるので、熱流束はどの点においても等しいと考えることができる。
\begin{equation}
	q=\frac{
		\sigma({T_1}^4 - {T_{s1}}^4)
	}{
		\frac{1}{\varepsilon_1} +
		\frac{1}{\varepsilon_{s1}} - 1
	}=\frac{
		\sigma({T_{s1}}^4 - {T_{s2}}^4)
	}{
		\frac{1}{\varepsilon_{s1}} +
		\frac{1}{\varepsilon_{s2}} - 1
	}=\frac{
		\sigma({T_{s2}}^4 - {T_{2}}^4)
	}{
		\frac{1}{\varepsilon_{s2}} +
		\frac{1}{\varepsilon_{2}} - 1
	}
\end{equation}

分母を払うと以下のようになる。

\begin{align}
	\frac{1}{\sigma}
	\left(
	\frac{1}{\varepsilon_1} +
	\frac{1}{\varepsilon_{s1}} - 1
	\right) q & =
	{T_1}^4 - {T_{s1}}^4    \\
	\frac{1}{\sigma}
	\left(
	\frac{1}{\varepsilon_{s1}} +
	\frac{1}{\varepsilon_{s2}} - 1
	\right) q & =
	{T_{s1}}^4 - {T_{s2}}^4 \\
	\frac{1}{\sigma}
	\left(
	\frac{1}{\varepsilon_{s2}} +
	\frac{1}{\varepsilon_{2}} - 1
	\right) q & =
	{T_{s2}}^4 - {T_{2}}^4
\end{align}

これらを足し合わせて、

\begin{equation}
	\frac{1}{\sigma}
	\left(
	\frac{1}{\varepsilon_1} +
	\frac{2}{\varepsilon_{s1}} +
	\frac{2}{\varepsilon_{s2}} +
	\frac{1}{\varepsilon_{2}} - 3
	\right) q =
	{T_1}^4 - {T_2}^4
\end{equation}

すなわち、

\begin{equation}
	q = \frac{
		\sigma({T_1}^4 - {T_2}^4)
	}{
		\frac{1}{\varepsilon_1} +
		\frac{2}{\varepsilon_{s1}} +
		\frac{2}{\varepsilon_{s2}} +
		\frac{1}{\varepsilon_{2}} - 3
	}
\end{equation}

これを計算すると、$q = 3.045\times10^3 \mathrm{\,W/m2}$となる。


\end{document}
