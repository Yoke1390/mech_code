\documentclass{jsarticle}
\usepackage{amsmath}
\usepackage[dvipdfmx]{graphicx}
\usepackage[dvipdfmx]{color}
\usepackage{mathtools}
\usepackage{physics}
\usepackage{siunitx}
\usepackage{wrapfig}
\usepackage{bm}
\mathtoolsset{showonlyrefs=true}

\begin{document}

\title{システム制御レポート}
\author{前田陽祐(03-240236)}
\date{2024-07-03}
\maketitle

\section{制御する対象のモデル化運動方程式から状態方程式を得る}
アクチュエータの力 \( U \) :
\[ U = \frac{d}{Ts + 1}I \]

マス・バネ・ダンパ系の運動方程式:
\[ m\ddot{z} + c\dot{z} + kz = U \]

以下のように状態変数を設定する。
\[ \begin{cases}
     {x}_1 = z \\
     {x}_2 = \dot{z} \\
     {x}_3 = U
\end{cases} \]

これにより、状態方程式は以下のようになる。

\[ \begin{cases}
\dot{x}_1 = x_2 \\
\dot{x}_2 = -\frac{k}{m}x_1 - \frac{c}{m}x_2 + \frac{1}{m}x_3 \\
\dot{x}_3 = \frac{1}{T}(dI - x_3)
\end{cases} \]

これを行列形式に変換する。
\[
\frac{d}{dt} \begin{bmatrix}
 x_1 \\\ x_2 \\\ x_3
\end{bmatrix}= \begin{bmatrix}
0 & 1 & 0 \\\ -\frac{k}{m} & -\frac{c}{m} & \frac{1}{m} \\\ 0 & 0 & -\frac{1}{T} \end{bmatrix}
\begin{bmatrix} x_1 \\\ x_2 \\\ x_3 \end{bmatrix}
+
\begin{bmatrix} 0 \\\ 0 \\\ \frac{d}{T} \end{bmatrix} I
\]

\section{制御目的の決定}
素早く安定化させることを目的とする

\section{制御方式}

MATLAB の lqr() 関数を使い、最適レギュレータを設計する。


\section{各状態の計測の問題}
位置 z のみ計測可能とする。

同一次元オブザーバに関する式は以下の通り。

\[
     \begin{cases}
          \dot{\hat{x}} = A\hat{x} + Bu + L(y - \hat{y}) \\
          \hat{y} = C\hat{x}
     \end{cases}
\]


\section{コントローラの設計各種パラメータ,極は各自が自由に設定して, コントローラゲイン,オブザーバゲインを求める}
\subsection{パラメータの設定}
例題の値よりも質量を大きくして、コントローラの電流のゲインを小さく設定する。
\[\begin{cases}
m = 2 \\
d = 0.5 \\
c = k = T = 1
\end{cases}\]

\subsection{MATLAB によるシミュレーション}
\begin{verbatim}
% パラメータ設定
m = 2;
k = 1;
c = 1;
d = 0.5;
T = 1;

% システム行列の定義
A = [0 1 0; 
     -k/m -c/m 1/m; 
     0 0 -1/T];
B = [0; 0; d/T];
C = [1 0 0; 0 1 0; 0 0 1];
D = [0; 0; 0];

% LQR法による最適制御器の設計
Q = C' * C;
R = 1;  % 重み行列(適宜調整)
[K, ~, ~] = lqr(A, B, Q, R);

% オブザーバゲインの設計
% オブザーバ行列の設計
poles = [-3; -4; -5];  % オブザーバ極(適宜調整)
L = place(A', C', poles)';

% 結果の表示
disp('制御ゲイン K:');
disp(K);

disp('オブザーバゲイン L:');
disp(L);
\end{verbatim}

\subsection{結果}
\begin{verbatim}
制御ゲイン K:
    0.0293    0.4286    0.4202

オブザーバゲイン L:
    3.0000    1.0000         0
   -0.5000    3.5000    0.5000
         0         0    4.0000
\end{verbatim}

\end{document}
