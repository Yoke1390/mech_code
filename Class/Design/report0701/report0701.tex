\documentclass{jsarticle}
\usepackage{amsmath}
\usepackage[dvipdfmx]{graphicx}
\usepackage[dvipdfmx]{color}
\usepackage{mathtools}
\usepackage{physics}
\usepackage{siunitx}
\usepackage{wrapfig}
\usepackage{bm}
\mathtoolsset{showonlyrefs=true}

\begin{document}

\title{条件付き最適設計}
\author{前田陽祐(03-240236)}
\maketitle

\section{目的}
講義における例を参考に、ある体積をもつ円筒容器の設計を考える。
その際、円筒容器に使用する材料の体積を最小化することを目的とする。

\subsection{設計変数}
円筒容器の内側の半径$r$と内側の高さ$h$、容器の厚み$t$を設計変数とする。

\subsection{制約関数}
円筒容器の体積$V$が一定であるとする。
\begin{equation}
  h(r, h) = \pi r^2 h - V = 0
\end{equation}

\subsection{目的関数}
円筒容器に使用する材料の体積$V_{\text{material}}$を最小化する。

材料力学より、円筒容器の内圧$p$に耐えられる容器の厚み$t$は内圧と外圧(大気圧)$p_0$の差$p-p_0$に比例する。
\begin{equation}
  t = \frac{(p-p_0) r}{\sigma}
\end{equation}
ここで、$\sigma$は材料の許容応力である。

また、内圧$p$の最大値は、容器の高さ$h$と大気圧$p_0$によって決まる。
\begin{equation}
  p = p_0 + \rho g h
\end{equation}
ここで、$\rho$は液体の密度、$g$は重力加速度である。
したがって、内圧と外圧の差は次のように表される。
\begin{equation}
  p - p_0 = \rho g h
\end{equation}

円筒容器の面積は$2\pi r^2 + 2\pi r h$であるので、材料の体積は次のように表される。
\begin{align}
  V_{\text{material}} &= (2\pi r^2 + 2\pi r h) t \\
                      &=  2\pi r (r+h) \frac{(p-p_0) r}{\sigma} \\
                      &= \frac{2\pi\rho g }{\sigma} r^2 h(r+h)
\end{align}

目的関数は定数倍を無視することができるので、次のように表される。
\begin{equation}
  f(r, h) = r^2 h (r+h) 
\end{equation}

\section{最適化}
\subsection{ラグランジュアン}
ラグランジュアン$L$は次のように表される。
\begin{align}
  L(r, h, \lambda) &= f(r, h) + \lambda h(r, h) \\
                   &= r^2 h (r+h) + \lambda (\pi r^2 h - V)
\end{align}

\subsection{必要条件}


\end{document}
