\documentclass{jsarticle}
\usepackage{amsmath}
\usepackage[dvipdfmx]{graphicx}
\usepackage[dvipdfmx]{color}
\usepackage{mathtools}
\usepackage{physics}
\usepackage{siunitx}
\usepackage{wrapfig}
\usepackage{bm}
\mathtoolsset{showonlyrefs=true}

\begin{document}

\title{条件付き最適設計}
\author{前田陽祐(03-240236)}
\maketitle

\section{目的}
講義における例を参考に、ある体積をもつ円筒容器の設計を考える。
その際、円筒容器に使用する材料の体積を最小化することを目的とする。

\subsection{設計変数}
円筒容器の内側の半径$r$と内側の高さ$h$、容器の厚み$t$を設計変数とする。

\subsection{制約関数}
円筒容器の体積$V$が一定であるとする。
\begin{equation}
  h(r, h) = \pi r^2 h - V = 0
\end{equation}

\subsection{目的関数}
円筒容器に使用する材料の体積$V_{\text{material}}$を最小化する。

材料力学より、円筒容器の内圧$p$に耐えられる容器の厚み$t$は内圧と外圧(大気圧)$p_0$の差$p-p_0$に比例する。
\begin{equation}
  t = \frac{(p-p_0) r}{\sigma}
\end{equation}
ここで、$\sigma$は材料の許容応力である。

また、内圧$p$の最大値は、容器の高さ$h$と大気圧$p_0$によって決まる。
\begin{equation}
  p = p_0 + \rho g h
\end{equation}
ここで、$\rho$は液体の密度、$g$は重力加速度である。
したがって、内圧と外圧の差は次のように表される。
\begin{equation}
  p - p_0 = \rho g h
\end{equation}

円筒容器の面積は$2\pi r^2 + 2\pi r h$であるので、材料の体積は次のように表される。
\begin{align}
  V_{\text{material}} &= (2\pi r^2 + 2\pi r h) t \\
                      &=  2\pi r (r+h) \frac{(p-p_0) r}{\sigma} \\
                      &= \frac{2\pi\rho g }{\sigma} r^2 h(r+h)
\end{align}

目的関数は定数倍を無視することができるので、次のように表される。
\begin{equation}
  f(r, h) = r^2 h (r+h)
  = r^3h + r^2h^2
\end{equation}

\section{最適化}
\subsection{ラグランジュアン}
ラグランジュアン$L$は次のように表される。
\begin{align}
  L(r, h, \lambda) &= f(r, h) + \lambda h(r, h) \\
                   &= r^3h + r^2h^2 + \lambda (\pi r^2 h - V)
\end{align}

\subsection{必要条件}
必要条件は$r, h, \lambda$のそれぞれでのラグランジュアンの偏微分が$0$になることである。
すなわち、
\begin{align}
  \pdv{L}{r}&= 3r^2h + 2rh^2 + 2\pi\lambda rh
  =0 \label{eq:d-r} \\
  \pdv{L}{h}&= r^3 + 2r^2h + \pi\lambda r^2
  =0 \label{eq:d-h} \\ 
  \pdv{L}{\lambda}&= \pi r^2h-V
  =0.\label{eq:d-lambda}
\end{align}

\subsection{解}
式\eqref{eq:d-lambda}より、
\begin{equation}
  h=\frac{V}{\pi r^2}
  \label{eq:h-r}
\end{equation}
が得られる。
これを式\eqref{eq:d-h}に代入すると、
\begin{align}
  r^3 + 2r^2\frac{V}{\pi r^2} + \pi\lambda r^2 = 0 \\
  r^3 + \frac{2V}{\pi} + \pi\lambda r^2 = 0 \\
  \lambda = -\frac{r}{\pi} - \frac{2V}{\pi^2 r^2}
  \label{eq:lambda-r}
\end{align}
となる。これを式\eqref{eq:d-r}に代入して、
\begin{align}
  3r^2\cdot\frac{V}{\pi r^2} 
  + 2r\cdot\qty(\frac{V}{\pi r^2})^2 
  + 2\pi\qty(
    -\frac{r}{\pi} - \frac{2V}{\pi^2 r^2}
    )
    \cdot r\cdot
    \frac{V}{\pi r^2} = 0 \\
  3\frac{V}{\pi} 
  + 2\frac{V^2}{\pi^2 r^3} 
  - 2\frac{V}{\pi}
  - 4\frac{V^2}{\pi^2 r^3} = 0 \\
  \frac{V}{\pi} - 2\frac{V^2}{\pi^2 r^3} = 0 \\
  r = \sqrt[3]{\frac{2V}{\pi}}.
\end{align}
したがって、式\eqref{eq:h-r}より、
\begin{equation}
  h = \frac{V}{\pi r^2} 
  = \sqrt[3]{\frac{V}{4\pi}}.
\end{equation}
式\eqref{eq:lambda-r}より、
\begin{equation}
  \lambda = -\frac{r}{\pi} - \frac{2V}{\pi^2 r^2} 
  = -\sqrt[3]{\frac{2V}{\pi^4}} - \sqrt[3]{\frac{2V}{\pi^4}} 
  = -\sqrt[3]{\frac{16V}{\pi^4}}.
\end{equation}
これが最適解の候補である。

\subsection{十分条件}
ヘッシアン行列$H$は以下のように表される。
\begin{align}
  H &=
  \mqty[
  \pdv[2]{L}{r} & \pdv{L}{r}{h} \\
  \pdv{L}{h}{r} & \pdv[2]{L}{h} \\
  ]
  \\&=
  \mqty[
  6rh+2h^2+2\pi\lambda h & 3r^2+4rh+2\pi\lambda r \\
  3r^2+4rh+2\pi\lambda r & 2r^2 \\
  ]
\end{align}
これに最適解の候補
\begin{equation}
  \left\{ \,
  \begin{aligned}
     & r = \sqrt[3]{\frac{2V}{\pi}} \\
     & h = \sqrt[3]{\frac{V}{4\pi}} \\
     & \lambda = -\sqrt[3]{\frac{16V}{\pi^4}}
  \end{aligned}
  \right.
  \label{eq:canditate}
\end{equation}
を代入すると、
\begin{align}
  H&=
  \mqty[
  6\sqrt[3]{\frac{V^2}{2\pi^2}} + 2\sqrt[3]{\frac{V^2}{16\pi^2}} - 2\pi\sqrt[3]{\frac{4V^2}{\pi^5}} &
  3\sqrt[3]{\frac{4V^2}{\pi^2}} + 4\sqrt[3]{\frac{V^2}{2\pi^2}} - 2\pi\sqrt[3]{\frac{4V^2}{\pi^5}} \\
  3\sqrt[3]{\frac{4V^2}{\pi^2}} + 4\sqrt[3]{\frac{V^2}{2\pi^2}} - 2\pi\sqrt[3]{\frac{4V^2}{\pi^5}} &
  2\sqrt[3]{\frac{4V^2}{\pi^2}}
  ]\\
  &= \sqrt[3]{\frac{V^2}{\pi^2}}
  \mqty[
  \sqrt[3]{108} + \sqrt[3]{\frac{1}{2}} - \sqrt[3]{32} &
  \sqrt[3]{108} + \sqrt[3]{32} - \sqrt[3]{32} \\
  \sqrt[3]{108} + \sqrt[3]{32} - \sqrt[3]{32} &
  \sqrt[3]{108}
  ]\\
  &=
  \mqty[
  \sqrt[3]{108} + \sqrt[3]{\frac{1}{2}} - \sqrt[3]{32} &
  \sqrt[3]{108} \\
  \sqrt[3]{108} &
  \sqrt[3]{108}
  ]
  \label{eq:H-val}
\end{align}
これを用いて、最適解の十分条件
$\partial{x}^*H\partial{x}>0 \text{ for }\forall\partial x\neq 0$
について考える。
\begin{align}
  \partial{x}^*H\partial{x}
  / \sqrt[3]{\frac{V^2}{\pi^2}}
  &= 
  \mqty[\partial{r} & \partial{h}]
  \mqty[
  \sqrt[3]{108} + \sqrt[3]{\frac{1}{2}} - \sqrt[3]{32} &
  \sqrt[3]{108} \\
  \sqrt[3]{108} &
  \sqrt[3]{108}
  ]
  \mqty[\partial{r} \\ \partial{h}] \\
  &=
  \qty(\sqrt[3]{108} + \sqrt[3]{\frac{1}{2}} - \sqrt[3]{32}) \cdot (\partial{r})^2
  + 2\qty(\sqrt[3]{108}) \cdot \partial{r}\partial{h}
  + \qty(\sqrt[3]{108}) \cdot (\partial{h})^2
\end{align}
ここで、$h=0$の条件より、以下の式を用いることができる。
\begin{equation}
  \nabla h \cdot \partial x = \mqty[\partial{r} & \partial{h}] \mqty[\partial{r} \\ \partial{h}] = 
  2\pi r h \partial{r} + \pi r^2 \partial{h} = 0
\end{equation}
これに最適解の候補\eqref{eq:canditate}を代入すると、
\begin{equation}
  2\pi r h \partial{r} + \pi r^2 \partial{h} 
  = 2\pi \sqrt[3]{\frac{V^2}{2\pi^2}} \partial{r} + \pi \sqrt[3]{\frac{4V^2}{\pi^2}} \partial{h}
\end{equation}
したがって、
\begin{equation}
  \partial{h} = -\partial{r}
\end{equation}
を得る。これを用いて、
\begin{align}
  \partial{x}^*H\partial{x}
  / \sqrt[3]{\frac{V^2}{\pi^2}}
  &=
  \qty(\sqrt[3]{108} + \sqrt[3]{\frac{1}{2}} - \sqrt[3]{32}) \cdot (\partial{r})^2
  + 2\qty(\sqrt[3]{108}) \cdot \partial{r}(-\partial{r})
  + \qty(\sqrt[3]{108}) \cdot (-\partial{r})^2 \\
  &=\qty(\sqrt[3]{\frac{1}{2}} - \sqrt[3]{32}) \cdot (\partial{r})^2 
  < 0
\end{align}
という結果になる。
したがって、最適解の候補\eqref{eq:canditate}は最適解ではないことがわかる。

\section{考察}
最適解の候補\eqref{eq:canditate}は最適解ではないことがわかった。
物理的な意味を考えると、$r\to 0$の極限では、円筒容器の体積が$0$になるため、最適解は$r\neq 0$であることがわかる。
\end{document}
